\pagebreak
\section{Other command-line arguments for instrumentation}
\label{sec:command-line-args}

This section describes a few additional flags that guide how
instrumentation is performed. 

\subsection{--sassi-debug}

This flag performs two functions.  First, it inserts canary
values on the stack before calling a handler function, then upon
return, it performs a check to make sure the canary values are
intact.  If the canary values differ from expectation, the check
traps.  This option is meant as a sanity check to help guard against
unknown stack corruption. (Because SASSI manipulates the
stack.) 

This flag can also be used to selectively disable some instrumentation
points.  When you assemble a PTX file with this option enabled, SASSI
will dump the SASS that was generated to file named
\texttt{sassi-instrumented-<PID>.txt}, and it will show you where the
instrumentation was inserted.  For example, look at this snippet from
the example code, with the branch example
(Section~\ref{sec:example2}):
\begin{verbatim}
      .
      .
      .
# BAR
# ==== BASIC BLOCK HEADER ====
# IADD
# IADD
# ISETP
+_Z13matrixMulCUDAILi16EEvPfS0_S0_ii, 111, _Z20sassi_before_handlerP17SASSIBeforeParams...
# BRA
# ==== BASIC BLOCK HEADER ====
# SYNC
      .
      .
      .
\end{verbatim}
The dump shows the instructions that were processed by SASSI,
including the ones around which SASSI added instrumentation.  The
lines that begin with ``+'', show where instrumentation was inserted.
The tuple on each ``+'' line shows the kernel, the unique function ID
of the instrumented instruction, and the handler that was called for
each instrumentation site.  So, the first ``+'' line shows that a call
to \texttt{sassi\_before\_handler} was inserted before the \texttt{BRA}
instruction.

If you take this output and put it in a file called
\texttt{sassi-skips.txt}, you can selectively disable instrumentation
for certain points by changing the ``+'' to a ``-''.  This selective
skipping of instrumentation points is only enabled in
\texttt{--sassi-debug} mode.  This simple interface has allowed us to
binary search for problems with SASSI's code generation.  But users
may find it useful, though tedious, for finer-grained control over
instrumentation sites.

\subsection{--sassi-inst-inline}

By default SASSI adds instrumentation via a code trampoline, which
better preserves the original code layout, i.e., the offset of each
instruction from the beginning of the function/kernel.  More
specifically, SASSI replaces the instrumented instruction with jump to
an instrumentation trampoline, which performs the instrumentation,
executes the instruction, and if necessary, branches back to the next
instruction to be executed. By setting this option to true, SASSI will
instead forgo a code trampoline, and inject the instrumentation
directly inline.

\subsection{--sassi-iff-true-predicate-handler-call}

NVIDIA GPUs support predication.  Most instructions can have a
boolean source argument that dictates whether or not that instruction
will actually execute.  If this predicate operand is false, should the
instrumentation handler be called?  By default, the handler is always
called, regardless of the predicate's value.  For ``before''
instrumentation, the handler can determine whether or not the
instruction \emph{will execute}; in the case of ``after''
instrumentation, the instrumentation handler can inspect the value of
the guarding predicate.

With this flag enabled, the handler will be called if and only
if the instruction is actually executed.

