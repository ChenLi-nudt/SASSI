\section{Prerequisites, Getting, and Installing SASSI}

SASSI is, in a nutshell, a specialized version of \texttt{ptxas} that
has a few extra options for instrumenting applications.  We distribute
this specialized \texttt{ptxas} as a (closed-source) binary blob using
GitHub's ``releases'' functionality.  However, to make SASSI more
usable, we also distribute several sample instrumentation libraries
and utilities that we have found to be useful.

Throughout the rest of the guide, we will refer to the following
environment variables:
\begin{itemize}
\item \texttt{\$SASSI\_REPO}: The base directory of the SASSI project,
  which you will clone from GitHub.
\item \texttt{\$CUDA\_HOME}: The location of your CUDA 7 installation
  (which is a prerequisite for installing SASSI).
\item \texttt{\$SASSI\_HOME}: The location of your SASSI installation,
  which is essentially a deep copy of your CUDA 7 installation, with a
  few added files and a different version of \texttt{ptxas}.
\end{itemize}

\subsection{Prerequisites} 

First, let's walk through SASSI's prerequisites.

\begin{enumerate}
\item {\bf Platform requirements}: SASSI requires an X86 64-bit host;
  a Fermi-, Kepler-, or Maxwell-based GPU; and at the time of this
  writing we have generated SASSI for Ubuntu (12 and 15), Debian
  7, and CentOS 6.\footnote{Please let us know if your platform
    is not listed, and we'll see what we can do.}
\item {\bf Install CUDA 7}: At the time of this writing, CUDA 7 can
  be fetched
  from:\\ \\  \texttt{https://developer.nvidia.com/cuda-downloads} \\
\item {\bf Make sure you have a 346.41 driver or newer:} The CUDA 7
  installation script can install a new driver for you that meets this
  requirement.  If you already have a newer driver, that should be
  fine.  You can test your driver version with the \texttt{nvidia-smi}
  command.
\item {\bf Get the SASSI project}: You can get the SASSI project from
  GitHub via the following command: \\ \\ \texttt{git clone
    https://github.com/NVlabs/SASSI} \\
  The project that you get will be your \texttt{\$SASSI\_REPO}.  The
  project contains several sample instrumentation libraries,
  documentation, and useful utilities.
\item {\bf Compilers [Optional]}: To use the instrumentation libraries
  we provide, you will need a C++-11 compiler.  We have been using GCC
  4.8.2.  For details about the host compiler that you can use on your
  platform for CUDA 7, consult the following
  guide:\\ \\ \texttt{http://docs.nvidia.com/cuda/cuda-getting-started-guide-for-linux/}
\end{enumerate}

\subsection{SASSI Project Structure}

The SASSI project that is hosted on GitHub has the directory structure
show in Figure~\ref{fig:structure}.  The \texttt{doc} directory
contains this document and an ISCA paper that describes SASSI's inner
workings.

The \texttt{example} directory contains a \texttt{Makefile} and a
single application from the CUDA SDK, which this guide uses for the
examples in Sections~\ref{sec:example1}-\ref{sec:example4}.

The \texttt{instlibs} directory contains sample SASSI instrumentation
libraries that range in complexity from very simple to fairly complex.
There are examples that profile branch behavior, memory behavior, and
operand value behavior.  In addition, there are a couple of
pedagogical examples that demonstrate SASSI's API.  We will continue
to add instrumentation libraries to this directory over time.  The
\texttt{instlibs/utils} directory contains a templatized device-side
hash map to facilitate bookkeeping.

\begin{figure*}
\center
\begin{verbatim}
$SASSI_REPO
  | EULA
  | FAQ
  | README
  | doc
    | sassi-user-guide.pdf
    | sassi-isca-2015.pdf
  | example
    | Makefile
    | matrixMul.cu 
  | instlibs
    | env.mk
    | LICENSE
    | Makefile
    | src
       | branch.cu
       | memdiverge.cu
       | ...
    | utils
       | sassi-dictionary.hpp
       | ...
\end{verbatim}
\caption{The directory structure of the SASSI project.}
\label{fig:structure}
\end{figure*}

\subsection{Installation}

We are using GitHub's ``releases'' functionality to release the
closed-source portion of SASSI.  Specifically, we are distributing
SASSI as a self-extracting installer that installs SASSI's specialized
version of \texttt{ptxas} and SASSI's header files that describe
SASSI's API.  To get a binary installer for your architecture, go to:\\
\begin{center}
\texttt{https://github.com/NVlabs/SASSI/releases}
\end{center}
Choose the installer for your architecture, download it, and execute it.
For example to install SASSI on an Ubuntu 15 x86-64 host, get the
installer, \texttt{SASSI\_x86\_64\_ubuntu\_vivid}, then execute it.

The installation script performs the following very simple tasks:
\begin{enumerate}
\item {\bf Has the end user agree to the End User License Agreement:}
  The license agreement is exactly the same as the EULA for the CUDA
  7 toolkit.
\item {\bf Asks the user for the location of the CUDA 7 toolkit:}
  This guide refers to this location as \texttt{\$CUDA\_HOME}.
\item {\bf Asks the user where it should install SASSI}: Note, this
  directory should be different from your CUDA 7 location.  Also note,
  you may need to run the script as root if you're planning on
  installing SASSI in the default location of
  \texttt{/usr/local/sassi}.  This guide refers to this location as
  \texttt{\$SASSI\_HOME}. 
\item {\bf Copies the whole CUDA 7 directory to the SASSI install
  location:} This step can take a while depending on the details of
  your host and filesystem.
\item {\bf Copies the SASSI header files to the SASSI install
  location.}  The header files are the best documentation for SASSI's
  interface, so you should know that they are installed in
  \lstinline[style=BashInputStyle]´$SASSI_HOME/include/sassi/´
\item {\bf Copies SASSI's \texttt{ptxas} to the SASSI install
  location.}  More specifically, SASSI's \texttt{ptxas} is moved to
  \lstinline[style=BashInputStyle]´$SASSI_HOME/bin/´
\end{enumerate}

Before going any further, you should make sure that the installation
script succesfully installed SASSI.  If there were no warnings,
errors, or abort messages, it probably worked.  Nevertheless, let's
check.  First, we'll make sure that the header files were installed in
the right place:
\begin{lstlisting}[style=BashInputStyle]
ls -l $SASSI_HOME/include/sassi
total 64 
-rw-r--r-- 1 mstephenson mstephenson  4759 Jul 17 08:05 sassi-branch.hpp
-rw-r--r-- 1 mstephenson mstephenson 11402 Jul 17 08:05 sassi-core.hpp
-rw-r--r-- 1 mstephenson mstephenson  2818 Jul 17 08:05 sassi.hpp
-rw-r--r-- 1 mstephenson mstephenson  3396 Jul 17 08:05 sassi-ins-properties.h
-rw-r--r-- 1 mstephenson mstephenson  3994 Jul 17 08:05 sassi-kernel.hpp
-rw-r--r-- 1 mstephenson mstephenson  4090 Jul 17 08:05 sassi-memory.hpp
-rw-r--r-- 1 mstephenson mstephenson  6499 Jul 17 08:05 sassi-opcodes.h
-rw-r--r-- 1 mstephenson mstephenson 10999 Jul 17 08:05 sassi-regs.hpp
-rw-r--r-- 1 mstephenson mstephenson  3636 Jul 17 08:05 sassi-types.h
\end{lstlisting}

Now we can check to see if the SASSI-enabled \texttt{ptxas} was
installed.  All SASSI instrumentation is turned off by default, so
without providing any SASSI-specific options, SASSI's \texttt{ptxas}
will behave like \texttt{ptxas} from your CUDA 7 installation.
Issuing \texttt{ptxas -h} will print out all options available,
including the optional arguments for instrumenting with SASSI.  Let's
see if the SASSI options are available:
\begin{lstlisting}[style=BashInputStyle]
$SASSI_HOME/bin/ptxas -h | grep sassi
--sassi-after-args <comma separated list of additional information> (-sassi-after-args)
--sassi-before-args <comma separated list of additional information> (-sassi-before-args)
--sassi-debug                                                        (-sassi-debug)
...
\end{lstlisting}

{\bf{Note:} If at any point you upgrade or patch your version of CUDA 7, you
  should re-run the installation script.}  The SASSI installation
script simply performs a deep copy of the CUDA 7 installation.

